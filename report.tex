\documentclass[10pt]{article}
\usepackage{graphicx}
\usepackage{amsmath}
\graphicspath{ {./} }
\usepackage{listings}
\lstset{
basicstyle=\small\ttfamily,
columns=flexible,
breaklines=true
}
\addtolength{\oddsidemargin}{-.875in}
	\addtolength{\evensidemargin}{-.875in}
	\addtolength{\textwidth}{1.75in}

	\addtolength{\topmargin}{-.875in}
	\addtolength{\textheight}{1.75in}
\renewcommand{\baselinestretch}{0.97}
\title{Assignment 2}
\author {Akshat Khare, 2016CS10315}

\date{March 11, 2019}
\begin{document}

\maketitle

\section{Text Classification: Naive Bayes}
\subsection{a}
Accuracy over test set: 60.9\% \\
Accuracy over training set: 64.4\% \\ 
\subsection{b}

\subsubsection{Random Accuracy}
Test data: 16.82\% \\ 
Train data: 16.69\% \\ 
\subsubsection{Most occuring based classification accurace}
Test data: 43.98\% \\
Train data: 43.88\% \\
\subsubsection{Improvement over Random based}
Test data: 44.08\% \\
Train data: 47.71\% \\
\subsubsection{Improvement over Most occuring based}
Test data: 16.92\% \\
Train data: 20.52\% \\
\subsection{c}
Test data confusion matrix: \\
\begin{lstlisting}
[[14494.  2867.  1371.  1094.  3081.]
 [ 3609.  3201.  1640.   655.   280.]
 [ 1149.  3368.  5283.  2459.   548.]
 [  496.  1039.  5344. 17712. 14067.]
 [  421.   363.   893.  7438. 40846.]]
\end{lstlisting}
Five stars rating has the most value (40846) of diagonal entry.\\
This means that five star is class which has most number of reviews which have been predicted correctly, i.e., it has most number of true positives.\\
We can see that one stars and five stars review have been classified most successfully as their diagonal entries are much more than corresponding non-diagonal entries.\\
Two stars and three stars have been poorly classified as their true positives can't overpower the false positives and true negatives. Four stars have mediocre performance.\\
So we can observe that five star reviews occur in majority and hence the previous observation of high most occurring based accuracy of 43.98\% occurs due to this reason. We can also observe that people vote majorly on extremes and the classification works good on extremes. Confusion matrix is hence a good measure of judging performance of the classifier.\\
\subsection{d}
Test data accuracy: 60.76\%\\
We see that applying stop word removal and stemming we in fact see a decrease in observed accuracy over test data.\\
Time taken significantly increased to 1.5 hour.\\
The confusion matrix obtained was:\\
\begin{lstlisting}
[[14522.  2980.  1484.  1197.  3063.]
 [ 3512.  3100.  1569.   613.   269.]
 [ 1073.  3121.  5089.  2406.   589.]
 [  514.  1157.  5265. 17110. 13474.]
 [  548.   480.  1124.  8032. 41427.]]
\end{lstlisting}
So we in fact observed a decrease in accuracy due reasons like presence of many words which were important to classify by stop words and changing of important words by stemming.\\
\subsection{e}
Applying bi-grams helping in increasing accuracy.\\
Test data accuracy: 63.97\%\\
Improvent over part $a$ is 3.07\%\\
Improvement over part $d$ is 3.21\%\\
So we can say that bigrams helped in improving overall accuracy.\\
The obtained confusion matrix was:\\
\begin{lstlisting}
[[16716.  4037.  1627.   736.  1201.]
 [  907.   956.   289.    63.    65.]
 [  750.  1935.  2115.   518.   160.]
 [ 1218.  3329.  9049. 17625.  9268.]
 [  578.   581.  1451. 10416. 48128.]]
\end{lstlisting}
So we can see that both 
%%% Fill in your content here.
\subsection{Keys and FDs}
%%% Fill in your content here.
\subsection{Sample Data}
%%% Fill in your content here.
\subsection{List of various relationships}
%%% Fill in your content here.

\section{Answer 2}
%%% Fill in your content here.

\section{Answer 3}
\subsection{Schema}
%%% Fill in your content here.
\subsection{Various insert modes}
%%% Fill in your content here.
\subsubsection{Bulk Load}
%%% Fill in your content here.
\subsubsection{Insert statements}
%%% Fill in your content here.
\subsubsection{using JDBC}
%%% Fill in your content here.
\subsection{Statistics}
%%% Fill in your content here.

\section{Answer 4}
%%% Fill in your content here.

\end{document}

